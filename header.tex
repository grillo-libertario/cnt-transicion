% https://tug.org/FontCatalogue/seriffonts.html > fuentes soportadas por latex
\renewcommand*{\familydefault}{\sfdefault}
\usepackage{pdfpages} % for the cover
\usepackage{sectsty}
\usepackage{wrapfig} % float figures
\usepackage{gmverse} % poetry, requires texlive-humanities https://osl.ugr.es/CTAN/macros/latex/contrib/gmverse/gmverse.pdf
\usepackage{changepage} % adjustwidth indented paragraph 
\usepackage{fancyhdr} % footer
\usepackage[all]{nowidow}
\pagestyle{fancy}
%\usepackage{graphicx} % images
% titlesec breaks hyperref, this is needed

% macro to force left page
\newcommand\cleartoleftpage{%
  \clearpage\ifodd\value{page}\begingroup\hbox{}\pagestyle{empty}\newpage\endgroup\fi
}

\newcommand\cleartorightpage{%
  \clearpage\ifodd\value{page}\else\begingroup\hbox{}\pagestyle{empty}\newpage\endgroup\fi
}

% macro for cover
\newcommand{\portada}[1]{%
    \includepdf[pages={1,{}}]{#1}%
}

% macro for category page
\newcommand{\portadaCategoria}[1]{%
  \begin{center}%
  \includegraphics[width=\textwidth]{#1}%
  \end{center}%
}

% macro for meal's photos
\newcommand{\fotoplato}[1]{%
\begin{wrapfigure}{r}{0.5\textwidth}%
  \centering
  \includegraphics[width=0.48\textwidth]{#1}%
\end{wrapfigure}%
}

% macro para el logo de la editora
\newcommand{\logoeditora}[1]{%
\begin{wrapfigure}{r}{0.15\textwidth}%
  \vspace{-20pt}
  \begin{center}
  \hspace{-40pt}
  \includegraphics[width=0.20\textwidth]{#1}%
  \end{center}
  \vspace{-10pt}
\end{wrapfigure}%
}

\newcommand{\innercover}[5]{%
\cleardoublepage
\thispagestyle{empty}
\null\vspace*{\fill}
\begin{center}
\LARGE\itshape
#1% Title
\end{center}
\par
\begin{center}
\Large\itshape
#2% Author
\end{center}
\par
\vspace*{\fill}
\begin{center}
#3% Prolog and others
\end{center}
\vspace*{\fill}
\begin{center}
\includegraphics[width=0.16\textwidth]{images/#4}% Editor/collection logo
\par
#5% Editor/Collection name
\end{center}
\clearpage
}

% macro for cc license
\newcommand{\cclicensetag}[1]{%
  \begin{center}%
    \href{http://creativecommons.org/licenses/#1/4.0/}{
      \includegraphics{images/#1}%
    }
  \end{center}%
}

% section formating for categories % https://www.overleaf.com/learn/latex/Font_sizes,_families,_and_styles
%\let\oldsection\section
%\let\oldsubsection\subsection
%\let\oldsubsubsection\subsubsection

%\makeatletter
%\renewcommand{\section}{%
%\@ifstar\oldsection\customsection%
%}
%\renewcommand{\subsection}{%
%\@ifstar\oldsubsection\customsubsection%
%}
%\makeatother
%\newcommand{\customsection}[1]{%
%% forces new page
%\cleartorightpage%
%% removes page in footers
%\thispagestyle{empty}%
%% verticall filler above
%\vspace*{\fill}%
%\oldsection{#1}%
%% double vertical filler below (places title in 1/3)
%\vspace*{\fill}%
%\vspace*{\fill}%
%% force page
%\cleartorightpage%
%}
%\newcommand{\customsubsection}[1]{%
%    \oldsubsection{#1}
%    \vspace*{3mm}
%}
%\renewcommand{\subsubsection}{\clearpage\oldsubsubsection}

\sectionfont{\clearpage\thispagestyle{inicioCapitulo}\color{Mahogany}\normalfont\LARGE\itshape}
\subsectionfont{\color{BlueViolet}\large}

% do not hyphenate
\hyphenpenalty=50000
\exhyphenpenalty=50000

% verse format
\versehangrightsquare
\stanzaskip 4mm
%\verseskipbefore 0mm
%\setlength{\verseskipafter}{-4mm}

% Redefinir verse com a poema treient l'espai 
\newenvironment*{poema}
{\parindent 0cm \parskip 0cm\verse}
{\endverse}

\fancypagestyle{inicioCapitulo}{%
\fancyhf{}% clear all header and footer fields
\renewcommand{\headrulewidth}{0pt}%
\renewcommand{\footrulewidth}{0pt}%
\fancyfoot[LE,RO]{\thepage}
}

\renewcommand{\sectionmark}[1]%
{\markboth{\MakeUppercase{#1}}{}}


\fancyhead[]{}
\fancyhead[RO]{\textsc{\nouppercase{\leftmark}}}
\fancyhead[LE]{\sc La CNT en la Transición }
\renewcommand{\headrulewidth}{0pt} % treu la linia de la capçalera que afegeix fancyhdr
\fancyfoot[CO,CE]{} % Elimina el footer que fica el pandoc al mig (Center, odd i even)
\fancyfoot[LE,RO]{\thepage}

% macro for fotos
% h! significa fuerza en el sitio (here)
% puedes poner diversos criterios ordenados por preferencia
% t (top), b (bottom), h (here), p (page, pagina dedicada), 
\newcommand{\fotoalancho}[2]{%
\begin{figure}[h!]
  \centering
  \includegraphics[width=0.9\textwidth]{images/#1}
  \caption{#2}
\end{figure}
}

